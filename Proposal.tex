%----------------------------------------------------------------------------------------
%	PACKAGES AND OTHER DOCUMENT CONFIGURATIONS
%----------------------------------------------------------------------------------------

\documentclass[12pt]{article}
\usepackage[english]{babel}
\usepackage[utf8x]{inputenc}
\usepackage{amsmath}
\usepackage{graphicx}
\usepackage[colorinlistoftodos]{todonotes}
\usepackage{ragged2e}
\usepackage[none]{hyphenat}
\usepackage[hidelinks]{hyperref}
\usepackage{listings}
\usepackage[title,titletoc,toc]{appendix}

\begin{document}

\begin{titlepage}

\newcommand{\HRule}{\rule{\linewidth}{0.5mm}} 
\center

%----------------------------------------------------------------------------------------
%	LOGO SECTION
%----------------------------------------------------------------------------------------

\includegraphics{images/uva.jpeg}\\[0.5cm]% Include a department/university logo - this will require the graphicx package
 
%----------------------------------------------------------------------------------------

%----------------------------------------------------------------------------------------
%	HEADING SECTIONS
%----------------------------------------------------------------------------------------
\textsc{\Large System and Network Engineering, MSc}\\[0.5cm] 
\textsc { \large Research Project 1}\\[0.4cm] % Title of your document

%----------------------------------------------------------------------------------------
%	TITLE SECTION
%----------------------------------------------------------------------------------------
\HRule \\[0.4cm]
{ \huge \bfseries Security Intelligence Data Mining}\\[0.4cm] % Title of your document
{ \large \bfseries Research Proposal}\\[0.4cm] % Title of your document
\HRule \\[0.4cm]




%----------------------------------------------------------------------------------------
%	AUTHOR SECTION
%----------------------------------------------------------------------------------------


\large Diana Rusu\\
{\bfseries Diana.Rusu@os3.nl}\\[0.5cm]
\large Nikolaos Petros Triantafyllidis\\
\bfseries Nikolaos.Triantafyllidis@os3.nl\\[2cm]

{\large \today} 

\end{titlepage}

\tableofcontents

\newpage

\section*{Introduction}
\addcontentsline{toc}{section}{Introduction}
\paragraph{}
With the increasing number of the cyber-attacks and malicious acts, security has became a big concern for different companies. Ensuring security for their infrastructure and being able to find threats in real time it's a crucial point. Data mining techniques can be used in discovering possible threats and future attacks from social websites. The primary points in this research will focus on the discovering and analysing security alerts. Techniques which are already available have been applied in different fields but this area still has futures to be discovered.

\section{Research Questions}
\paragraph{}

'How the reliable sources related to security attacks can be defined?'

From this main research question the following secondary question arise. 

How can be obtained and extracted the primary content?
How are the keywords ascertained?
Process, analyse the resulted data, how? (define the patterns and models,training set)


\section{Related work}

Several studies have been conducted in the past in the field of social web mining. One recent book that has been released by  O'Reilly Media has as it's main purpose to  "discover who’s making connections with whom, what they’re talking about, and where they’re located?" \cite{first}. The social websites that are analysed include Facebook, Twitter, LinkedIn, Google+, GitHub and others. This book comes along with a virtual machine which includes toolbox that can be experimented \cite{second}.


\section{Scope, what is in/out of scope given limitations}

The main purpose of this project is to define and implement a security intelligence architecture that will be able to extract and analyse possible attacks to a related company. 

 
\section{Methodology}


\section{Experimental installation requirements}

To facilitate a proper investigation in the allocated time, several open tools will be used and tested. As a development environment two different languages will be used and combined namely Go and Java. This research does not depend on the Operating System used as both systems are platform independent. Collected data will be dropped in data base like MongoDB. For this purpose depending on how large the data will be, hardware space will be needed.

\section{Ethical implications}

This research will focus and mine the data from the public sources. If any sensitive information or security attacks will be encountered during the analysing of this data, targeted companies and our supervisors will be informed. We do not intend to keep or collect any personal information. All the data gathered will be kept no longer then our research project period. 


\bibliographystyle{plain}

\begin{thebibliography}{99}
\bibitem{second}
  GitHub, (2014). ptwobrussell/Mining-the-Social-Web-2nd-Edition. [online] Available at: https://github.com/ptwobrussell/Mining-the-Social-Web-2nd-Edition [Accessed 6 Jan. 2015].
  \bibitem{first}
  Russell, M. (2015). Mining the Social Web. [online] Shop.oreilly.com. Available at: http://shop.oreilly.com/product/0636920030195.do [Accessed 6 Jan. 2015].
  \bibitem{some:try3}
  I am trying this 3
\end{thebibliography}

\end{document}