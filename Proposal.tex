%----------------------------------------------------------------------------------------
%	PACKAGES AND OTHER DOCUMENT CONFIGURATIONS
%----------------------------------------------------------------------------------------

\documentclass[12pt]{article}
\usepackage[english]{babel}
\usepackage[utf8x]{inputenc}
\usepackage{amsmath}
\usepackage{graphicx}
\usepackage[colorinlistoftodos]{todonotes}
\usepackage{ragged2e}
\usepackage[none]{hyphenat}
\usepackage[hidelinks]{hyperref}
\usepackage{listings}
\usepackage[title,titletoc,toc]{appendix}

\begin{document}

\begin{titlepage}

\newcommand{\HRule}{\rule{\linewidth}{0.5mm}} 
\center

%----------------------------------------------------------------------------------------
%	LOGO SECTION
%----------------------------------------------------------------------------------------

\includegraphics{images/uva.jpeg}\\[0.5cm]% Include a department/university logo - this will require the graphicx package
 
%----------------------------------------------------------------------------------------

%----------------------------------------------------------------------------------------
%	HEADING SECTIONS
%----------------------------------------------------------------------------------------
\textsc{\Large System and Network Engineering, MSc}\\[0.5cm] 
\textsc { \large Research Project 1}\\[0.4cm] % Title of your document

%----------------------------------------------------------------------------------------
%	TITLE SECTION
%----------------------------------------------------------------------------------------
\HRule \\[0.4cm]
{ \huge \bfseries Security Intelligence Data Mining}\\[0.4cm] % Title of your document
{ \large \bfseries Research Proposal}\\[0.4cm] % Title of your document
\HRule \\[0.4cm]




%----------------------------------------------------------------------------------------
%	AUTHOR SECTION
%----------------------------------------------------------------------------------------


\large Diana Rusu\\
{\bfseries Diana.Rusu@os3.nl}\\[0.5cm]
\large Nikolaos Petros Triantafyllidis\\
\bfseries Nikolaos.Triantafyllidis@os3.nl\\[2cm]

{\large \today} 

\end{titlepage}

\tableofcontents

\newpage

\section*{Introduction}
\addcontentsline{toc}{section}{Introduction}
With the increasing number of cyber-attacks and the growth of computer crime worldwide, it becomes apparent that IT security is a major concern and crucial survival factor for large companies, organisations and institutions of any sort. Security Operations departments working to ensure confidentiality, integrity and availability for the system infrastructure of their organisation, invest huge parts of their time and effort in detecting threats in real time. A very valuable source of security intelligence, vital to cyber-risk assessment, is information mined from data posted on public sites such as "pastebins" or social networks. However, this is a very cumbersome task due to the lack of Natural Language Processing capabilities in most of the existing tools. Moreover, as recent events have showcased, several threats arise from governments and criminal associations originating from countries whose languages use non-latin scripts (Chinese, Russian, Korean, etc.). It is, hence, important to have data mining tools that provide support for such alphabets and languages, since a lot of a security intelligence can be discovered in such texts. The main goals of this research project will be to explore the various public data and detect the most appropriate among them. Moreover, numerous current data analytics techniques as well as their application on security related issues will be assessed. Lastly the above knowledge will be applied on the implementation of a simple system that will work as a proof-of-concept and help determine the technical feasibility, storage requirements and operational cost of such a system.  

\section{Research Questions}
This topic is admittedly very open and several specific research questions can be defined some of which we will try to answer to some extent. The main questions on which we will be focusing can be the following:

\begin{enumerate}
	\item How can the raw data be effectively collected from the public sources? 
	\begin{itemize}
		\item How can we effectively detect the reliable sources?
		\item What search terms can we deploy during the retrieval phase?
		\item How can the unstructured data be pre-processed? 
	\end{itemize}
	\item How can the data be analysed in respect to security operations?
	\begin{itemize}
		\item How can we apply current Data Mining and Analytics techniques on Security issues?
		\item How can we derive the risk assessment model from the above?
		\item How can we apply the model on new data?
	\end{itemize}
	\item	How can the collected knowledge be applied on a system implementation?
	\begin{itemize}
		\item What is a reliable and extensible System Architecture that can be designed?
		\item What are the computational and storage requirements of such a system?
		\item What extensions can be proposed for that system?
	\end{itemize}
\end{enumerate}

%'How the reliable sources related to security attacks can be defined?'

%From this main research question the following secondary question arise. 

%How can be obtained and extracted the primary content?
%How are the keywords ascertained?
%Process, analyse the resulted data, how? (define the patterns and models,training set)


\section{Related work}
\paragraph{}
Several research studies have been conducted in the past in the field of social web mining. One recent book that has been released by  O'Reilly Media has as it's main purpose exploring and mining social websites (e.g., Facebook, Twitter, LinkedIn, Google+, GitHub and others). It goes from the premise of "discovering who’s making connections with whom, what they’re talking about, and where they’re located?" \cite{first}. This book comes along with a virtual machine which includes toolbox that can be experimented on\cite{second}. Different new methods and techniques are introduced for a further investigation on a specific source. Intrusion detection\cite{intrusion-detection} has been a big concern and special attention has been provided to this field. Integrity, confidentiality and available are one of the main purposes for computer security.  One system\cite{third} has been proposed to police and uses data mining techniques to  detect and predict possible act of crimes and frauds. 


\section{Scope, what is in/out of scope given limitations}

The main purpose of this project is to define and implement a security intelligence architecture that will be able to extract and analyse possible attacks to a related company. 

 
\section{Methodology}


\section{Experimental installation requirements}

To facilitate a proper investigation in the allocated time, several open tools will be used and tested. As a development environment two different languages will be used and combined namely Go and Java. This research does not depend on the Operating System used as both systems are platform independent. Collected data will be dropped in data base like MongoDB. For this purpose depending on how large the data will be, hardware space will be needed.

\section{Ethical implications}

This research will focus and mine the data from the public sources. If any sensitive information or security attacks will be encountered during the analysing of this data, targeted companies and our supervisors will be informed. We do not intend to keep or collect any personal information. All the data gathered will be kept no longer then our research project period. 


\bibliographystyle{plain}

\begin{thebibliography}{99}
\bibitem{first}
  Russell, M. (2015). Mining the Social Web. [online] Shop.oreilly.com. Available at: http://shop.oreilly.com/product/0636920030195.do [Accessed 6 Jan. 2015].
\bibitem{second}
  GitHub, (2014). ptwobrussell/Mining-the-Social-Web-2nd-Edition. [online] Available at: https://github.com/ptwobrussell/Mining-the-Social-Web-2nd-Edition [Accessed 6 Jan. 2015].
 \bibitem{third}
Anon, (2015). [online] Available at: http://www.sentient.nl/docs/data\_mining\_for\_intelligence\_led\_policing.pdf [Accessed 7 Jan. 2015].
\bibitem{intrusion-detection}
Anon, (2015). [online] Available at: https://www.usenix.org/legacy/publications/library/proceedings/sec98/full\_papers/lee/lee.pdf [Accessed 7 Jan. 2015].
  \bibitem{some:try3}
  I am trying this 3
\end{thebibliography}

\end{document}